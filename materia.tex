\begin{multicols}{2}
    \section{Inducción}
    \subsubsection*{Principio del Buen Orden (PBO)}
    $$
    A \neq \varnothing, A \subseteq \mathbb{N} \Rightarrow \exists x \in A \text{ tal que } \forall y \in A, x \leq y
    $$
    
    \subsubsection*{Principio de Inducción Simple (PIS)}
    \begin{itemize}
        \item Primera formulación:
        Si
        \begin{enumerate}
            \item $0 \in A$ (base inductiva).
            \item Si $n \in A$ (hipótesis inductiva), entonces $n + 1 \in A$ (tesis inductiva).
        \end{enumerate}
        entonces $A = \mathbb{N}$.
        \item Segunda formulación:
        Si
        \begin{enumerate}
            \item $P(0)$ es verdadero
            \item Si $P(n)$, entonces $P(n + 1)$ (cada vez que $n$ cumple $P$, $n + 1$ también la cumple)
        \end{enumerate}
        entonces todos los elementos de $\mathbb{N}$ cumplen la propiedad P.
        \item Tercera formulación:
        Si
        \begin{enumerate}
            \item $P(\mathbf{n_0})$ es verdadero
            \item Si $P(n)$, entonces $P(n + 1)$ (cada vez que $n$ cumple $P$, $n + 1$ también la cumple)
        \end{enumerate}
        entonces todos los elementos de $\mathbb{N}$ \textbf{a partir de} $\mathbf{n_0}$ cumplen la propiedad P.
    \end{itemize}
    
    \subsubsection*{Principio de Inducción por Curso de Valores (PICV)}
    \begin{itemize}
        \item Primera formulación:
        Si
        $$
        \{0,1,\ldots,n - 1\} \subseteq A \Rightarrow n \in A
        $$
        entonces $A = \mathbb{N}$.
        \item Segunda formulación:
        Si
        \begin{align*}
        \forall k \in \mathbb{N}, k < n, P(k) \text{ es verdadero } \\
        \Rightarrow P(n)
        \text{ es verdadero}
        \end{align*}
        entonces $P$ es verdadero para todos los elementos de $\mathbb{N}$.
    \end{itemize}
    
    \subsubsection*{Teorema 1}
    Los 3 principios de inducción (PBO, PIS y PICV) son equivalentes.
    
    \subsubsection*{Conjunto definido inductivamente}
    \begin{enumerate}
        \item El conjunto es el menor que cumple las reglas.
        \item Definir conjunto de elementos base.
        \item Definir conjunto finito de reglas de construcción de nuevos elementos en base a los elementos iniciales.
    \end{enumerate}
    
    \subsubsection*{Inducción estructural}
    Si
    \begin{enumerate}
        \item Todos los elementos base de $A$ (conjunto definido inductivamente) cumplen la propiedad $P$
        \item Para cada regla de construcción, si la regla se aplica sobre elementos en $A$ que cumplen la propiedad $P$, entonces los elementos producidos por la regla también cumplen la propiedad $P$
    \end{enumerate}
    entonces todos los elementos en $A$ cumplen la propiedad $P$.
    
    \section{Lógica proposicional}
    \subsubsection*{Sintaxis}
    Sea $P$ un conjunto de variables proposicionales. El conjunto de todas las \textbf{fórmulas} de lógica proposicional sobre $P$, denotado por $L(P)$, es el menor conjunto que cumple las siguientes reglas:
    \begin{enumerate}
        \item Si $p \in P$, entonces $p \in L(P)$.
        \item Si $\alpha \in L(P)$, entonces $(\neg \alpha) \in L(P)$.
        \item Si $\alpha, \beta \in L(P)$, entonces $(\alpha \wedge \beta ) \in L(P)$, $(\alpha \vee \beta) \in L(P)$, $(\alpha \rightarrow \beta) \in L(P)$ y $(\alpha \leftrightarrow \beta) \in L(P)$.
    \end{enumerate}
    
    \subsubsection*{Semántica}
    Una \textbf{valuación} o \textbf{asignación de verdad} para las variables proposicionales en un conjunto $P$ es una función $\sigma: P \rightarrow \{0,1\}$, donde $0 =$ falso y $1 = $ verdadero.
    
    \subsubsection*{Tablas de verdad}
    \begin{multicols}{2}
        \begin{table}[H]
            \begin{tabular}{c|c}
                $p$ & $\neg p$ \\ \hline
                0   & 1        \\
                1   & 0
            \end{tabular}
        \end{table}
    
        \begin{table}[H]
            \begin{tabular}{cc|c}
                $p$ & $q$ & $p \wedge q$ \\ \hline
                0   & 0   & 0            \\
                0   & 1   & 0            \\
                1   & 0   & 0            \\
                1   & 1   & 1
            \end{tabular}
        \end{table}
    
        \begin{table}[H]
            \begin{tabular}{cc|c}
                $p$ & $q$ & $p \vee q$ \\ \hline
                0   & 0   & 0          \\
                0   & 1   & 1          \\
                1   & 0   & 1          \\
                1   & 1   & 1
            \end{tabular}
        \end{table}
    
        \begin{table}[H]
            \begin{tabular}{cc|c}
                $p$ & $q$ & $p \rightarrow q$ \\ \hline
                0   & 0   & 1                 \\
                0   & 1   & 1                 \\
                1   & 0   & 0                 \\
                1   & 1   & 1
            \end{tabular}
        \end{table}
    
        \begin{table}[H]
            \begin{tabular}{cc|c}
                $p$ & $q$ & $p \leftrightarrow q$ \\ \hline
                0   & 0   & 1                     \\
                0   & 1   & 0                     \\
                1   & 0   & 0                     \\
                1   & 1   & 1
            \end{tabular}
        \end{table}
    
        \begin{table}[H]
            \begin{tabular}{cc|c}
                $p$ & $q$ & XOR \\ \hline
                0   & 0   & 0   \\
                0   & 1   & 1   \\
                1   & 0   & 1   \\
                1   & 1   & 0
            \end{tabular}
        \end{table}
    \end{multicols}
    
    \subsubsection*{Equivalencia lógica $\equiv$}
    Dos fórmulas $\alpha, \beta \in L(P)$ son \textbf{lógicamente equivalentes} (denotado como $\alpha \equiv \beta$) si para toda valuación $\sigma$ se tiene que $\sigma(\alpha) = \sigma(\beta)$.
    
    \subsubsection*{Leyes de equivalencia}
        \begin{enumerate}
            \item Doble negación
    
                  $\neg(\neg \alpha) \equiv \alpha$
    
            \item De Morgan
    
                  $\neg(\alpha \wedge \beta) \equiv (\neg \alpha) \vee (\neg \beta)$
    
                  $\neg(\alpha \vee \beta) \equiv (\neg \alpha) \wedge (\neg \beta)$
    
            \item Conmutatividad
    
                  $\alpha \wedge \beta \equiv \beta \wedge \alpha$
    
                  $\alpha \vee \beta \equiv \beta \vee \alpha$
    
            \item Asociatividad
    
                  $\alpha \wedge (\beta \wedge \gamma) \equiv (\alpha \wedge \beta) \wedge \gamma$
    
                  $\alpha \vee (\beta \vee \gamma) \equiv (\alpha \vee \beta) \vee \gamma$
    
            \item Distributividad
    
                  $\alpha \wedge (\beta \vee \gamma) \equiv (\alpha \wedge \beta) \vee (\alpha \wedge \gamma)$
    
                  $\alpha \vee (\beta \wedge \gamma) \equiv (\alpha \vee \beta) \wedge (\alpha \vee \gamma)$
    
            \item Idempotencia
    
                  $\alpha \wedge \alpha \equiv \alpha$
                  \vspace{-4px}
    
                  $\alpha \vee \alpha \equiv \alpha$
    
            \item Absorción
    
                  $\alpha \wedge (\alpha \vee \beta) \equiv \alpha$
    
                  $\alpha \vee (\alpha \wedge \beta) \equiv \alpha$
    
            \item Implicancia
    
                  $\alpha \rightarrow \beta \equiv (\neg \alpha) \vee \beta$
    
            \item Doble implicancia
    
                  $\alpha \leftrightarrow \beta \equiv (\alpha \rightarrow \beta) \wedge (\beta \rightarrow \alpha)$
        \end{enumerate}
    
    \subsubsection*{Operadores generalizados}
    \begin{align*}
        \bigwedge_{i=1}^{n} \alpha_i & = \alpha_1 \wedge \alpha_2 \wedge \ldots \wedge \alpha_n \\
        \bigvee_{i=1}^{n} \alpha_i   & = \alpha_1 \vee \alpha_2 \vee \ldots \vee \alpha_n
    \end{align*}
    
    \subsubsection*{Teorema 2}
    Podemos representar cualquier tabla de verdad con una fórmula. Ademas, podemos representar cualquier tabla de verdad con una fórmula que sólo usa $\neg$, $\wedge$ y $\vee$.
    
    \subsubsection*{Conectivos funcionalmente completos}
    Un conjunto de conectivos se dice así si toda fórmula en $L(P)$ es lógicamente equivalente a una fórmula que sólo usa esos conectivos. Ejemplos:
    \begin{multicols}{2}
    \begin{itemize}
        \item $\{\neg, \wedge, \vee\}$
        \item $\{\neg, \wedge\}$
        \item $\{\neg, \vee\}$
        \item $\{\neg, \rightarrow\}$
    \end{itemize}
    \end{multicols}
    
    \subsubsection*{Satisfacibilidad}
    Una fórmula $\alpha$ es \textbf{satisfacible} si existe una valuación $\sigma$ tal que $\sigma(\alpha) = 1$.
    
    \subsubsection*{Contradicción}
    Una fórmula $\alpha$ es una \textbf{contradicción} si no es satisfacible; es decir, para toda valuación $\sigma$ se tiene que $\sigma(\alpha) = 0$.
    
    \subsubsection*{Tautología}
    Una fórmula $\alpha$ es una \textbf{tautología} si para toda valuación $\sigma$ se tiene que $\sigma(\alpha) = 1$.
    
    \subsubsection*{Teorema 3}
    Dos fórmulas $\alpha, \beta \in L(P)$ son \textbf{lógicamente equivalentes} si $\alpha \leftrightarrow \beta$ es una \textbf{tautología}.
    
    \subsubsection*{Forma Normal Disyuntiva (DNF)}
    Una fórmula $\alpha$ está en DNF si es una disyunción de conjunciones de literales (variable proposicional o su negación); osea, si es de la forma
    $$
        B_1 \vee B_2 \vee \ldots \vee B_k
    $$
    donde cada $B_i$ es una conjunción de literales, $B_i = (l_{i1} \wedge \ldots \wedge l_{ik_i})$.
    
    \subsubsection*{Forma Normal Conjuntiva (CNF)}
    Una fórmula $\alpha$ está en CNF si es una conjunción de disyunciones de literales; osea, si es de la forma
    $$
        C_1 \wedge C_2 \wedge \ldots \wedge C_k
    $$
    donde cada $C_i$ es una disyunción de literales, $C_i = (l_{i1} \vee \ldots \vee l_{ik_i})$.
    
    \begin{itemize}
        \item Una disyunción de literales se llama \textbf{cláusula}.
    \end{itemize}
    
    \subsubsection*{Teorema 4}
    Toda fórmula es equivalente a una fórmula en DNF.
    
    \subsubsection*{Teorema 5}
    Toda fórmula es equivalente a una fórmula en CNF.
    
    \subsubsection*{Propiedad formas normales}
    Toda fórmula $\alpha \in L(P)$ en DNF con a lo más $n$ disyunciones es lógicamente equivalente a una fórmula $\beta$ en CNF ($\alpha \equiv \beta$).
    
    \subsubsection*{Consecuencia lógica $\models$}
    Una fórmula $\alpha$ es consecuencia lógica de $\Sigma$ si para cada valuación $\sigma$ tal que $\sigma(\Sigma) = 1$, se tiene que $\sigma(\alpha) =  1$. Se denota como $\Sigma \models \alpha$.
    
    \subsubsection*{Consecuencias lógicas clásicas}
        \begin{itemize}
            \item Modus ponens
    
                  $\{p, p \rightarrow q\} \models q$
    
            \item Modus tollens
    
                  $\{\neg q, p \rightarrow q\} \models \neg p$
    
            \item Demostración por partes
    
                  $\{p \vee q \vee r, p \rightarrow s, q \rightarrow s, r \rightarrow s\} \models s$
    
            \item Resolución
    
                  $\{p \vee q, \neg q \vee r\} \models p \vee r$
        \end{itemize}
    
    \subsubsection*{Conjunto satisfacible}
    Un conjunto de fórmulas $\Sigma$ es \textbf{satisfacible} si existe una valuación $\sigma$ tal que $\sigma(\Sigma) = 1$. En caso contrario, $\Sigma$ es \textbf{inconsistente}.
    
    \subsubsection*{Teorema 6}
    $\Sigma \models \alpha$ si y sólo si $\Sigma \cup \{\neg \alpha\}$ es inconsistente.
    
    \subsubsection*{Teorema 7}
    Un conjunto de fórmulas $\Sigma$ es inconsistente si y sólo si $\Sigma \models \square$, con $\square$ una cláusula vacía.
    
    \subsubsection*{Equivalencia lógica para conjuntos}
    Dos conjuntos de fórmulas $\Sigma_1$ y $\Sigma_2$ son \textbf{lógicamente equivalentes} ($\Sigma_1 \equiv \Sigma_2$) si para toda valuación $\sigma$ se tiene que $\sigma(\Sigma_1) = \sigma(\Sigma_2)$. También diremos que $\Sigma$ es lógicamente equivalente a una fórmula $\alpha$ si $\Sigma \equiv {\alpha}$.
    
    \subsubsection*{Conjuntos y conjunción}
    Todo conjunto de fórmulas $\Sigma$ es equivalente a la conjunción de sus fórmulas.
    $$
        \Sigma \equiv \bigwedge_{\alpha \in \Sigma} \alpha
    $$
    
    \subsubsection*{Teorema 8}
    Todo conjunto de fórmulas $\Sigma$ es equivalente a un conjunto de cláusulas.
    
    \subsubsection*{Resolución para consecuencia lógica}
    Para resolver el problema de consecuencia lógica (es decir, determinar si $\Sigma \models \alpha$) tenemos que determinar si un conjunto de cláusulas $\Sigma'$ construido desde $\Sigma \cup \{\neg \alpha\}$ es tal que $\Sigma' \models \square$.
    
    \subsubsection*{Resolución proposicional}
    \begin{multicols}{2}
    Regla de resolución \p
    
                  \begin{tabular}{c}
                      $C_1 \vee l \vee C_2$            \\
                      $C_3 \vee \overline{l} \vee C_2$ \\ \hline
                      $C_1 \vee C_2 \vee C_3 \vee C_4$
                  \end{tabular}
    
    Regla de factorización \p
    
                  \begin{tabular}{c}
                      $C_1 \vee l \vee C_2 \vee l \vee C_3$ \\ \hline
                      $C_1 \vee l \vee C_2 \vee C_3$
                  \end{tabular}
    \end{multicols}
    Dado un conjunto $\Sigma$ de cláusulas, una \textbf{demostración por resolución} de que $\Sigma$ es inconsistente es una secuencia de cláusulas $C_1,\ldots,C_n$ tal que $C_n = \square$ y para cada $i = 1\ldots n$ se tiene que
    
              \begin{itemize}
                  \item $C_i \in \Sigma$ o
                  \item $C_i$ se obtiene de dos cláusulas anteriores en la secuencia usando la regla de resolución, o
                  \item $C_i$ se obtiene de una cláusula anterior en la secuencia usando la regla de factorización.
              \end{itemize}
    
              Si existe tal demostración, escribimos $\Sigma \vdash \square$.
    
    \subsubsection*{Teorema 8}
    Dado un conjunto de cláusulas $\Sigma$, se tiene que: \p
        
    \textbf{Correctitud:} Si $\Sigma \vdash \square$, entonces $\Sigma \models \square$ ($\Sigma$ es inconsistente).
    
    \textbf{Completitud:} Si $\Sigma \models \square$, entonces $\Sigma \vdash \square$.
    \begin{itemize}
        \item \textbf{Corolario 1:} Si $\Sigma$ es un conjunto de cláusulas, entonces $\Sigma \models \square$ si y sólo si $\Sigma \vdash \square$. Dicho de otra manera, un conjunto de cláusulas $\Sigma$ es inconsistente si y sólo si existe una demostración por resolución de que es inconsistente.
        \item \textbf{Corolario 2:} Dados un conjunto de fórmulas $\Sigma$ y una fórmula $\alpha$ cualesquiera,
        $$
        \Sigma \models \alpha \text{ si y sólo si } \Sigma' \vdash \square
        $$
        donde $\Sigma'$ es un conjunto de cláusulas tal que $\Sigma \cup \{\neg \alpha\} \equiv \Sigma'$.
    \end{itemize}
    
    \section{Lógica de predicados}
    \subsubsection*{Predicado}
    Un predicado $n$-ario $P(x_1,\ldots,x_n)$ es una afirmación con $n$ variables, cuyo valor de verdad depende de los objetos en el cual es evaluado.
    \begin{itemize}
        \item $P$ es el \textbf{símbolo} del predicado.
        \item Todos los predicados están restringidos a un \textbf{dominio} de evaluación.
        \item Para un predicado $P(x_1,\ldots,x_n)$ diremos que $x_1,\ldots,x_n$ son \textbf{variables libres} de $P$.
        \item Un predicado \textbf{0-ario} es un predicado sin variables y tiene valor de verdadero o falso sin importar la valuación.
    \end{itemize}
    
    \subsubsection*{Cuantificador universal $\forall$}
    Sea $P(x, y_1, \ldots, y_n)$ un predicado compuesto con dominio $D$. Definimos el cuantificador universal:
        $$
        P'(y_1, \ldots, y_n) = \forall x(P(x, y_1, \ldots, y_n))
        $$
        donde $x$ es la variable cuantificada y $y_1, \ldots, y_n$ son las variables libres. Para $b_1, \ldots, b_n$ en $D$, definimos la valuación:
        $$
        P'(b_1, \ldots, b_n) = 1
        $$
        si \textbf{para todo} $a$ en $D$ se tiene que $P(a, b_1, \ldots, b_n) = 1$, y 0 en otro caso.
    
    \subsubsection*{Cuantificador existencial $\exists$}
    Sea $P(x, y_1, \ldots, y_n)$ un predicado compuesto con dominio $D$. Definimos el cuantificador existencial:
        $$
        P'(y_1, \ldots, y_n) = \exists x(P(x, y_1, \ldots, y_n))
        $$
        donde $x$ es la variable cuantificada y $y_1, \ldots, y_n$ son las variables libres. Para $b_1, \ldots, b_n$ en $D$, definimos la valuación:
        $$
        P'(b_1, \ldots, b_n) = 1
        $$
        si \textbf{existe} $a$ en $D$ tal que $P(a, b_1, \ldots, b_n) = 1$, y 0 en otro caso.
    
    \subsubsection*{Interpretaciones $\mathcal{I}$}
    Una interpretación $\mathcal{I}$ para $P_1, \ldots, P_n$ está compuesta de:
        \begin{itemize}
            \item un dominio $D$ que denotaremos $\mathcal{I}(dom)$ y
            \item un predicado $P_i^D$ que denotaremos por $\mathcal{I}(P_i)$ para cada símbolo $P_i$.
        \end{itemize}
    
    \subsubsection*{Satisfacibilidad para interpretaciones}
        $$
        \mathcal{I} \models \alpha(a_1,\ldots,a_n)
        $$
        si $\alpha(a_1,\ldots,a_n)$ es verdadero al interpretar cada símbolo en $\alpha$ según $\mathcal{I}$. Si $\mathcal{I}$ \textbf{no satisface} $\alpha$ sobre $a_1,\ldots,a_n$ en $\mathcal{I}(dom)$ lo denotamos como $\mathcal{I} \nvDash \alpha(a_1,\ldots,a_n)$.
    
    \subsubsection*{Equivalencia lógica para predicados}
    Sean $\alpha(x_1, \ldots, x_n)$ y $\beta(x_1,\ldots,x_n)$ dos fórmulas en lógica de predicados. Decimos que $\alpha$ y $\beta$ son \textbf{lógicamente equivalentes}:
    $$
    \alpha \equiv \beta
    $$
    si para toda interpretación $\mathcal{I}$ y para todo $a_1,\ldots,a_n$ en $\mathcal{I}(dom)$ se cumple:
    $$
    \mathcal{I} \models \alpha(a_1,\ldots,a_n) \text{ si y sólo si } \mathcal{I} \models \beta(a_1,\ldots,a_n)
    $$
    \begin{itemize}
        \item \textbf{Caso especial:} Si $\alpha$ y $\beta$ son oraciones (no tienen variables libres), entonces:
        $$
        \mathcal{I} \models \alpha \text{ si y sólo si } \mathcal{I} \models \beta
        $$
    \end{itemize}
    
    \subsubsection*{Teorema 9}
    Sea $\alpha(x)$, $\beta(x)$ fórmulas con $x$ su variable libre, entonces:
    \begin{align*}
        \neg \forall x(\alpha(x)) &\equiv \exists x(\neg \alpha(x))\\
        \neg \exists x(\alpha(x)) &\equiv \forall x(\neg \alpha(x))\\
        \forall x(\alpha(x) \wedge \beta(x)) &\equiv \forall x(\alpha(x)) \wedge \forall x (\beta(x))\\
        \exists x(\alpha(x) \vee \beta(x)) &\equiv \exists x(\alpha(x)) \vee \exists x(\beta(x))    
    \end{align*}
    
    \subsubsection*{Consecuencia lógica para predicados}
    Una fórmula $\alpha$ es consecuencia lógica de un conjunto de fórmulas $\Sigma$:
        $$
        \Sigma \models \alpha
        $$
        si para toda interpretación $\mathcal{I}$ y $a_1,\ldots,a_n$ en $\mathcal{I}(dom)$ se cumple que:
        $$
        \text{si } \mathcal{I} \models \Sigma(a_1,\ldots, a_n) \text{ entonces } \mathcal{I} \models \alpha(a_1,\ldots,a_n)
        $$
    
    \subsubsection*{Reglas de inferencia para predicados}   
    
    \begin{itemize}
    \item Especificación universal
    \vspace{5px}
    
        \begin{tabular}{c}
            $\forall x (\alpha(x))$ \\ \hline
            $\alpha(a)$ para cualquier $a$
        \end{tabular}
    
    \item Generalización universal
    \vspace{5px}
    
    
    \begin{tabular}{c}
        $\alpha(a)$ para un $a$ arbitrario \\ \hline
        $\forall x (\alpha(x))$
    \end{tabular}
    
    \item Especificación existencial
    \vspace{5px}
    
    
    \begin{tabular}{c}
        $\exists x (\alpha(x))$ \\ \hline
        $\alpha(a)$ para algún $a$ (nuevo)
    \end{tabular}
    
    \item Generalización existencial
    \vspace{5px}
    
    \begin{tabular}{c}
        $\alpha(a)$ para algún $a$ \\ \hline
        $\exists x (\alpha(x))$
    \end{tabular}
    \end{itemize}
    
    \section{Demostraciones}
    \subsubsection*{Método directo}
    Por demostrar: $\forall x(P(x) \rightarrow Q(x))$. \p
    
    Suponemos que $P(n)$ es verdadero para un $n$ arbitrario y demostramos que $Q(n)$ también es verdadero.
    
    \subsubsection*{Contrapositivo}
    Por demostrar: $\forall x(P(x) \rightarrow Q(x))$. \p
    
    Suponemos que $Q(n)$ es falso para un $n$ arbitrario y demostramos que $P(n)$ también es falso.
    
    \subsubsection*{Contradicción}
    Por demostrar: $\forall x(P(x) \rightarrow Q(x))$. \p
    
    Suponemos que existe un $n$ tal que $P(n)$ es verdadero y $Q(n)$ es falso e inferimos una contradicción.
    
    \subsubsection*{Por casos}
    Para cada subdominio $C_1,\ldots,C_n$ demostramos que:
    $$
    \forall x(P(x) \rightarrow Q(x))
    $$
    \begin{itemize}
        \item Aquí, dividimos el dominio de la interpretación $\mathcal{I}$ con que trabajamos en una cantidad finita de casos $C_1,\ldots,C_n$, tal que:
        $$
        \mathcal{I}(dom) = \bigcup_{i=1}^{n}C_i
        $$
    \end{itemize}
    
    \subsubsection*{Doble implicación}
    Por demostrar: $\forall x (P(x) \leftrightarrow Q(x))$. \p
    
    Aquí, se deben demostrar ambas direcciones por separado. En términos formales:
    $$
    \forall x(P(x) \rightarrow Q(x)) \wedge \forall x(Q(x) \rightarrow P(x))
    $$
    
    \subsubsection*{Contra-ejemplo}
    Por demostrar: $\neg (\forall x(P(x)))$. \p
    
    Aquí, hay que encontrar un elemento $n$ (cualquiera) tal que $P(n)$ es falso.
    
    \subsubsection*{Existencial}
    Por demostrar: $\exists x (P(x))$.\p
    
    Aquí, debemos demostrar que existe un elemento $n$ tal que $P(n)$ es verdadero (nótese que NO es estrictamente necesario mostrar $n$ explícitamente).
    
    \section{Teoría de conjuntos}
    \subsubsection*{Conjunto}
    Es una colección \textit{bien definida} de objetos. Estos objetos se llaman \textbf{elementos} del conjunto, y diremos que \textbf{pertenecen} a él.
    
    \subsubsection*{Subconjunto}
    Sean $A$ y $B$ conjuntos. Diremos que $A$ es \textbf{subconjunto} de $B$, denotado por $A \subseteq B$, si
    $$
    \forall x (x \in A \rightarrow x \in B)
    $$
    En otras palabras, $A$ es subconjunto de $B$ si cada elemento de $A$ está tambien en $B$.
    
    \subsubsection*{Igualdad de conjuntos}
    Dos conjuntos $A$ y $B$ son iguales si y sólo si $A \subseteq B$ y $B \subseteq A$. Otra definición equivalente es
    $$
    \forall A \forall B \quad A = B \leftrightarrow \forall x (x \in A \leftrightarrow x \in B)
    $$
    \begin{itemize}
        \item Este es el \textbf{axioma de extensión}.
        \item \textbf{Observación:} Los conjuntos no pueden tener elementos repetidos. $\{x,x\} = \{x\}$.
    \end{itemize}
    
    \subsubsection*{Subconjunto propio}
    Diremos que $A$ es un subconjunto propio de $B$, detonado por $A \subsetneq B$, si
    $$
    A \subseteq B \wedge A \neq B \text{, o alternativamente, } A \subseteq B \wedge B \nsubseteq A
    $$
    
    \begin{itemize}
        \item \textbf{Corolario:} $B \nsubseteq A$ si y sólo si $\exists x \in B$ tal que $x \notin A$.
    \end{itemize}
    
    \subsubsection*{Axioma del conjunto vacío}
    $\exists X$ tal que $\forall x, x \notin X$. Lo denotaremos por $\varnothing$ o \{\}.
    
    \subsubsection*{Teorema 10}
    Para todo conjunto $A$ se tiene que $\varnothing \subseteq A$.
    
    \subsubsection*{Teorema 11}
    Existe un único conjunto vacío.
    
    \subsubsection*{Formas de definir un conjunto}
    \begin{itemize}
        \item Por extensión (listando sus elementos): 
        $$\mathbb{Z}_5 = \{0,1,2,3,4\}$$
        \item Por comprensión: 
        $$\mathbb{Z}_5 = \{x \,|\, x \in \mathbb{N} \wedge x < 5\}$$
    \end{itemize}
    
    \subsubsection*{Axioma de abstracción}
    Si $\alpha$ es una propiedad sobre objetos, entonces $A = \{x \,|\, \alpha(x)\}$ es un conjunto.
    
    \subsubsection*{Axioma de separación}
    Si $\alpha$ es una propiedad y $C$ es un conjunto ``sano", entonces $A = \{x \,|\, x \in C \wedge \alpha(x)\}$ es un conjunto.
    \begin{itemize}
        \item Un \textbf{conjunto sano} es uno el cual no fue creado usando el axioma de abstracción.
    \end{itemize}
    
    \subsubsection*{Operaciones}
    \begin{itemize}
        \item Unión: $A \cup B = \{x \,|\, x \in A \vee x \in B\}$
        \item Intersección: $A \cap B = \{x \,|\, x \in A \wedge x \in B\}$
        \item Diferencia: $A \backslash B = \{x \,|\, x \in A \wedge x \notin B\}$
        \item Conjunto potencia: $\mathcal{P}(A) = \{X \,|\, X \subseteq A\}$ \p
        
        Algunas observaciones:
        \begin{multicols}{2}
            \begin{itemize}
                \item $\varnothing \in \mathcal{P}(A)$
                \item $A \in \mathcal{P(A)}$
                \item $A \subseteq A \cup B$
                \item $A \cap B \subseteq A$
            \end{itemize}
        \end{multicols}
    \end{itemize}
    
    \subsubsection*{Complemento}
    Sea $A \subseteq \mathcal{U}$ (siendo $\mathcal{U}$ un conjunto universal fijo) un conjunto cualquiera. El complemento de $A$ (relativo a $\mathcal{U}$) es el conjunto
        $$
        A^c = \mathcal{U} \backslash A = \{x \,|\, x \in \mathcal{U} \wedge x \notin A\}
        $$
    
    \subsubsection*{Teorema 12}
    Si $A$, $B$ y $C$ son conjuntos cualquiera (subconjuntos de $\mathcal{U}$), entonces se cumplen las siguientes leyes:    
    \begin{itemize}
        \item Asociatividad
        
        $A \cup (B \cup C) = (A \cup B) \cup C$
    
        $A \cap (B \cap C) = (A \cap B) \cap C$
    
        \item Conmutatividad
        
        $A \cup B = B \cup A$
    
        $A \cap B = B \cap A$
    
        \item Idempotencia
        
        $A \cup A = A$
    
        $A \cap A = A$
    
        \item Absorción
        
        $A \cup (A \cap B) = A$
    
        $A \cap (A \cup B) = A$
    
        \item Elemento neutro
        
        $A \cup \varnothing = A$
    
        $A \cap \mathcal{U} = A$
    
        \item Distributividad
        
        $A \cup (B \cap C) = (A \cup B) \cap (A \cup C)$
    
        $A \cap (B \cup C) = (A \cap B) \cup (A \cap C)$
    
        \item Leyes de De Morgan
        
        $(A \cup B)^c = A^c \cap B^c$
    
        $(A \cap B)^c = A^c \cup B^c$
    
        \item Elemento inverso
        
        $A \cup A^c = \mathcal{U}$
    
        $A \cap A^c = \varnothing$
    
        \item Dominación
        
        $A \cup \mathcal{U} = \mathcal{U}$
    
        $A \cap \varnothing = \varnothing$
        
    \end{itemize}
    
    \subsubsection*{Unión generalizada}
    Siendo $\mathcal{S}$ un conjunto de conjuntos, la unión generalizada representa a la unión de todos los conjuntos componentes de $\mathcal{S}$; es decir, contiene a todos los elementos que pertenecen a algún conjunto de $\mathcal{S}$.
    $$
    \bigcup \mathcal{S} = \{x \,|\, \exists A \in \mathcal{S} \text{ tal que } x \in A\} = \bigcup_{A \in \mathcal{S}} A
    $$
    
    \subsubsection*{Intersección generalizada}
    Siendo $\mathcal{S}$ un conjunto de conjuntos, la intersección generalizada representa a la intersección de todos los conjuntos componentes de $\mathcal{S}$; es decir, contiene a todos los elementos que pertenecen a todos los conjuntos de $\mathcal{S}$.
    $$
    \bigcap \mathcal{S} = \{x \,|\, \forall A \in \mathcal{S} \text{ se cumple que } x \in A\} = \bigcap_{A \in \mathcal{S}} A
    $$
    
    \section{Relaciones}
    \subsubsection*{Par ordenado}
    Sean $a,b \in \mathcal{U}$ (donde $\mathcal{U}$ es un conjunto universal). Definimos el par ordenado $(a,b)$ como
    $$
    (a,b) = \{\{a\}, \{a,b\}\}
    $$
    \begin{itemize}
        \item Propiedad: $(a,b) = (c,d)$ si y sólo si $a = c \wedge b = d$
    \end{itemize}
    
    \subsubsection*{$n$-tupla}
    Sean $a_1, \ldots,a_n \in \mathcal{U}$. Definimos una $n$-tupla como:
    $$
    (a_1, \ldots, a_n) = ((a_1, \ldots, a_{n-1}), a_n)
    $$
    
    \subsubsection*{Producto cartesiano}
    Dados dos conjuntos $A$ y $B$, definimos el producto cartesiano entre $A$ y $B$ como
    $$
    A \times B = \{(a,b) \,|\, a \in A \wedge b \in B\}
    $$
    Se puede extender esta noción a más conjuntos. Dados conjuntos $A_1, \ldots, A_n$, definimos el producto cartesiano entre los $A_i$ como
    $$
    A_1 \times \ldots \times A_n = \{(a_1, \ldots, a_n) \,|\, a_1 \in A_1 \wedge \ldots \wedge a_n \in A_n\}
    $$
    
    \subsubsection*{Relación}
    Dados conjuntos $A_1, \ldots, A_n$, diremos que $R$ es una relación sobre tales conjuntos si $R \subseteq A_1 \times \ldots \times A_n$.
            \begin{itemize}
                \item La \textbf{aridad} de una relación $R$ es el tamaño de las tuplas que la componen. Así, diremos que $R$ es una relación $n$-aria.
            \end{itemize}
    
    \subsubsection*{Relación binaria}
    Dados conjuntos $A$ y $B$, diremos que $R$ es una relación binaria de $A$ en $B$ si $R \subseteq A \times B$. 
    \begin{itemize}
        \item Podemos tener una relación sobre un solo conjunto, es decir, dado un conjunto $A$, diremos que $R$ es una relación binaria sobre $A$ si $R \subseteq A \times A = A^2$.
        \item Cuando tengamos productos cartesianos entre un mismo conjunto, usaremos una notación de \textit{potencia}:
        $$
        A \times \overset{(n-2 veces)}{\cdots} \times = A^n
        $$
    \end{itemize}
    
    \subsubsection*{Relación binaria: Divide a $|$}
    Esta relación sobre los naturales sin el 0, es una tal que $a$ está relacionado con $b$ si y sólo si $b$ es múltiplo de a:
    $$
    a\,|\,b \text{ si y sólo si } \exists k \in \mathbb{N} \text{ tal que } b = ka
    $$
    
    \subsubsection*{Relación binaria: Equivalencia módulo $n$ $\equiv_n$}
    Esta relación sobre los naturales, es una tal que $a$ está relacionado con $b$ si y sólo si $|a - b|$ es múltiplo de $n$:
    $$
    a \equiv_n b \text{ si y sólo si } \exists k \in \mathbb{N} \text{ tal que } |a - b| = kn
    $$
    
    \subsubsection*{Propiedades de las relaciones}
    Una relación es
    \begin{itemize}
        \item \textbf{Refleja} si para cada $a \in A$ se tiene que $R(a,a)$.
        \item \textbf{Irrefleja} si para cada $a \in A$ \textit{no} se tiene que $R(a,a)$. 
        \item \textbf{Simétrica} si para cada $a,b \in A$, si $R(a,b)$ entonces $R(b,a)$.
        \item \textbf{Asimétrica} si para cada $a,b \in A$, si $R(a,b)$ entonces \textit{no es cierto} que $R(b,a)$.
        \item \textbf{Antisimétrica} si para cada $a,b \in A$, si $R(a,b)$ y $R(b,a)$, entonces $a = b$.
        \item \textbf{Transitiva} si para cada $a,b,c \in A$, si $R(a,b)$ y $R(b,c)$, entonces $R(a, c)$.
        \item \textbf{Conexa} si para cada $a,b \in A$, se tiene que $R(a,b)$ o $R(b,a)$. \p
        
        En lógica de predicados (para demostrar):
    \end{itemize}
    
    \begin{itemize}
        \item Refleja: $\forall x(R(x, x))$
        \item Irrefleja: $\forall x(\neg R(x,x))$
        \item Simétrica: $\forall x \forall y (R(x,y) \rightarrow R(y,x))$
        \item Asimétrica: $\forall x \forall y (R(x,y) \rightarrow \neg R(y,x))$
        \item Antisimétrica: $\forall x \forall y ((R(x,y) \wedge R(y,x)) \rightarrow x = y)$
        \item Transitiva: $\forall x \forall y \forall z ((R(x,y) \wedge R(y,z)) \rightarrow R(x, z))$
        \item Conexa: $\forall x \forall y (R(a,b) \vee R(b,a))$
    \end{itemize}
    
    \subsubsection*{Relaciones de equivalencia}
    Una relación $R$ sobre $A$ es una relación de equivalencia si es refleja, simétrica y transitiva.
    
    \subsubsection*{Clase de equivalencia}
    Sea $\thicksim$ una relación de equivalencia sobre un conjunto $A$ y un elemento $x \in A$. La clase de equivalencia de $x$ bajo $\thicksim$ es el conjunto
    $$
    [x]_\thicksim = \{y \in A \,|\, x \thicksim y\}
    $$
    
    \subsubsection*{Teorema 13}
    Sea $\thicksim$ una relación de equivalencia sobre un conjunto $A$. Entonces:
    \begin{enumerate}
        \item $\forall x \in A, x \in [x]$
        \item $x \thicksim y$ si y sólo si $[x] = [y]$
        \item Si $[x] \neq [y]$ entonces $[x] \cap [y] = \varnothing$
    \end{enumerate}
    
    \subsubsection*{Conjunto cuociente}
    Sea $\thicksim$ una relación de equivalencia sobre un conjunto $A$. El conjunto cuociente de $A$ con respecto a $\thicksim$ es el conjunto de todas las clases de equivalencia de $\thicksim$:
    $$
    A/\thicksim = \{[x] \,|\, x \in A\}
    $$
    
    \subsubsection*{Índice}
    Es la cantidad de clases de equivalencia que induce. Es decir, la cantidad de elementos de su conjunto cuociente.
    
    \subsubsection*{Partición}
    Sea $A$ un conjunto cualquiera, y $S$ una colección de subconjuntos de $A$ ($S \subseteq \mathcal{P}(A)$). Diremos que $S$ es una partición de $A$ si cumple que:
    \begin{enumerate}
        \item $\forall X \in S, X \neq \varnothing$
        \item $\bigcup S = A$
        \item $\forall X, Y \in S$ si $X \neq Y$ entonces $X \cap Y = \varnothing$
    \end{enumerate}
    
    \subsubsection*{Teorema 14}
    Si $\thicksim$ es una relación de equivalencia sobre un conjunto $A$, entonces $A/\thicksim$ es una partición de $A$.
    
    \subsubsection*{Relación $\downarrow$} 
    Esta relación sobre $\mathbb{N} \times \mathbb{N}$ se define como:
    $$
    (m,n) \downarrow (r,s) \quad \leftrightarrow \quad m + s = n + r
    $$
    
    \subsubsection*{Números enteros}
    El conjunto de los números $\mathbb{Z}$ se define como el conjunto cuociente de $\mathbb{N}^2$ respecto a $\downarrow$:
    $$
    \mathbb{Z} = \mathbb{N}/\downarrow = \{[(0,0)],[(0,1)],[(1,0)], [(0,2)],[(2,0)],\ldots\}
    $$
    \begin{itemize}
        \item $[(0,0)]$ será el entero 0.
        \item $[(0,i)]$ sera el entero $i$.
        \item $[(i,0)]$ será el entero $-i$.
    \end{itemize}
    \textbf{Nota:} ``$-1$'' es sólo un \textbf{nombre} para la clase de equivalencia $[(1,0)]$. El símbolo ``$-$'' no significa nada por sí solo.
    
    \subsubsection*{Suma sobre $\mathbb{Z}$} 
    $[(m,n)] +_\downarrow [(r,s)] = [(m + r, n +s)]$
    
    \subsubsection*{Producto sobre $\mathbb{Z}$} $[(m,n)] \cdot_\downarrow [(r,s)] = [(m \cdot s + n \cdot r, m \cdot r + n \cdot s)]$
    
    \section{Funciones}
    
    \subsubsection*{Función}
    Sea $f$ una relación binaria de $A$ en $B$; es decir, $f \subseteq A \times B$. Diremos que $f$ es una función de $A$ en $B$ si dado cualquier elemento $a \in A$, si existe un elemento $b \in B$ tal que $afb$, este es único:
    $$
    afb \wedge afc \rightarrow b = c
    $$
    Si $afb$, escribimos $b = f(a)$.
    \begin{itemize}
        \item $b$ es la \textbf{imagen} de $a$.
        \item $a$ es la \textbf{preimagen} de b.
        \item \textbf{Notación:} $f: A \rightarrow B$
    \end{itemize}
    
    \subsubsection*{Función total}
    Una función $f:A \rightarrow B$ se dice total si todo elemento en $A$ tiene imagen, es decir
    \begin{itemize}
        \item Para todo $a \in A$ existe $b \in B$ tal que $b = f(a)$.
        \item Una función que no sea total se dice \textbf{parcial}.
        \item Toda función será total a menos que se diga lo contrario.
    \end{itemize}
    
    \subsubsection*{Inyectividad}
    Una función $f:A \rightarrow B$ es inyectiva (o $1-1$) si para cada par de elementos $x,y \in A$ se tiene que $f(x) = f(y) \rightarrow x = y$. Es decir, no existen dos elementos distintos en $A$ con la misma imagen.
    
    \subsubsection*{Sobreyectividad}
    Una función $f:A \rightarrow B$ es sobreyectiva (o sobre) si cada elemento $b \in B$ tiene preimagen. Es decir, para todo $b \in B$ existe un $a \in A$ tal que $b = f(a)$.
    
    \subsubsection*{Biyectividad}
    Una función $f:A \rightarrow B$ es biyectiva si es inyectiva y sobreyectiva \textbf{a la vez}.
    
    \subsubsection*{Relación inversa}
    Dada una relación $R$ de $A$ en $B$, una relación inversa de $R$ es una relación de $B$ en $A$ definida como
    $$
    R^{-1} = \{ (b,a) \in B \times A \,|\, aRb \}
    $$
    
    \subsubsection*{Función invertible}
    Dada una función $f$ de $A$ en $B$, diremos que $f$ es invertible si su relación inversa $f^{-1}$ es una relación de $B$ en $A$.
    
    \subsubsection*{Composición de relaciones}
    Dadas relaciones $R$ de $A$ en $B$ y $S$ de $B$ en $C$, la composición de $R$ y $S$ es una relación de $A$ en $C$ definida como
    $$
    S \circ R = \{ (a,c) \in A \times C \,|\, \exists b \in B \text{ tal que } aRb \wedge bSc \}
    $$
    
    \subsubsection*{Composición de funciones}
    Dadas funciones $f$ de $A$ en $B$ y $g$ de $B$ en $C$, la composición $g \circ f$ es una función de $A$ en $C$.
    
    \subsubsection*{Teorema 14}
    Si $f: A \rightarrow B$ es biyectiva, entonces la relación inversa $f^{-1}$ es una función biyectiva de $B$ en $A$.
    \begin{itemize}
        \item Si $f$ es biyectiva, entonces es invertible.
    \end{itemize}
    
    \subsubsection*{Teorema 15}
    Dados dos funciones $f: A \rightarrow B$ y $g: B \rightarrow C$:
    \begin{itemize}
        \item Si $f$ y $g$ son inyectivas, entonces $g \circ f$ también lo es.
        \item Si $f$ y $g$ son sobreyectivas, entonces $g \circ f$ también lo es.
    \end{itemize}
    
    \subsubsection*{Principio del palomar}
    Se tienen $m$ palomas y $n$ palomares, con $m > n$. Entonces, si se reparten las $m$ palomas en los $n$ palomares, necesariamente existirá un palomar con más de una paloma.
    
    \subsubsection*{Principio del palomar (matemático)}
    Si se tiene una función $f: \mathbb{N}_m \rightarrow \mathbb{N}_n$, con $m > n$, la función $f$ no puede ser inyectiva. Es decir, necesariamente existirán $x,y \in \mathbb{N}_m$ tales que $x \neq y$, pero $f(x) = f(y)$.
    
    \subsubsection*{Principio del palomar (sobreyectividad)}
    Si se tiene una función $f: \mathbb{N}_m \rightarrow \mathbb{N}_n$, con $m < n$, la función $f$ no puede ser sobreyectiva.
    
    \subsubsection*{Conjunto equinumeroso}
    Sean $A$ y $B$ dos conjuntos cualesquiera. Diremos que $A$ es equinumeroso con $B$ (o que $A$ tiene el mismo tamaño que $B$) si existe una función biyectiva $f: A \rightarrow B$. Lo denotamos como
    $$
    A \approx B
    $$
    Esta es una \textbf{relación de equivalencia}.
    
    \subsubsection*{Cardinalidad}
    La cardinalidad de un conjunto $A$ es su clase de equivalencia bajo $\approx$:
    $$
    |A| = [A]_\approx
    $$
    
    \subsubsection*{Conjunto finito}
    Diremos que $A$ es un conjunto finito si $A \approx n$, para algún $n \in \mathbb{N}$. Es decir, si existe una función biyectiva $f: A \rightarrow n = \{0,\ldots, n - 1\}$.
    \begin{itemize}
        \item En tal caso, se tiene que $|A| = [n]_\approx$.
        \item Por simplicidad, diremos que $|A| = n$.
        \item También podemos decir que $A$ tiene $n$ elementos.
    \end{itemize}
    
    \subsubsection*{Lema 1}
    Sean $A$ y $B$ dos conjuntos finitos tales que $A \cap B = \varnothing$. Entonces, $|A \cup B| = |A| + |B|$.
    
    \subsubsection*{Teorema 16}
    Sea $A$ un conjunto finito. Entonces, se cumple que $|P(A)| = 2^{|A|}$.
    \begin{itemize}
        \item Esto implica que si $A$ es un conjunto finito, entonces su cardinalidad es \textbf{estrictamente menor} que la de su conjunto potencia.
    \end{itemize}
    
    \subsubsection*{Conjunto enumerable}
    Un conjunto $A$ se dice enumerable si $|A| = |\mathbb{N}|$.
    
    \subsubsection*{Teorema 17 (Schröder-Bernstein)}
    $A \approx B$ si y sólo si existen funciones inyectivas $f: A \rightarrow B$ y $g: B \rightarrow A$.
    
    \subsubsection*{Conjunto enumerable (alternativa)}
    Un conjunto $A$ es enumerable si y sólo si todos sus elementos se pueden poner en una lista infinita; es decir, si existe una sucesión infinita
    $$
    (a_0, a_1, a_2, \ldots, a_n, a_{n+1}, \ldots)
    $$
    tal que \textit{todos} los elementos de $A$ aparecen en la sucesión \textit{una única vez} cada uno.
    
    \subsubsection*{Teorema 18 (Cantor)}
    El intervalo real $(0,1) \subseteq \mathbb{R}$ es infinito pero no enumerable.
    
    \subsubsection*{Teorema 19}
    $$
    (0, 1) \approx \mathbb{R} \approx \mathcal{P}(\mathbb{N})
    $$
    
    \subsubsection*{Relación $\preceq$ entre conjuntos}
    Dados conjuntos $A$ y $B$, diremos que $A \preceq B$ ($A$ no es más grande que $B$), si existe una función inyectiva $f: A \rightarrow B$.
    \begin{itemize}
        \item $\preceq$ \textbf{no} es una relación de orden.
        \item Si $A \preceq B$, diremos que $|A| \leq |B|$.
    \end{itemize}
    
    \subsubsection*{Conjunto menos numeroso}
    Dados conjuntos $A$ y $B$, diremos que $A \prec B$ ($A$ es menos numeroso que $B$) si $A \preceq B$ pero $A \not\approx$ B.
    \begin{itemize}
        \item Esta noción con funciones se define de tal forma que existe una función inyectiva $f: A \rightarrow B$, pero no existe una función biyectiva $g: A \rightarrow B$.
        \item Si $A \prec B$, diremos que $|A| < |B|$.
        \item Corolario: $|\mathbb{N}| < |\mathcal{P}(\mathbb{N})|$.
    \end{itemize}
    
    \subsubsection*{Teorema 20 (Cantor, menos numeroso)}
    $A$ es menos numeroso que su conjunto potencia, es decir
    $$
    |A| < |\mathcal{P}(A)|
    $$
    
    \section{Análisis de algoritmos}
    
    \subsubsection*{Algoritmo}
    Es un método para convertir un \textbf{input} en un \textbf{output}. A estos métodos les exigiremos ciertas propiedades:
    \begin{itemize}
        \item \textbf{Precisión:} cada instrucción debe ser planteada de forma precisa y no ambigua.
        \item \textbf{Determinismo:} cada instrucción tiene un único comportamiento que depende sólo del input.
        \item \textbf{Finitud:} el algoritmo está compuesto por un conjunto finito de instrucciones.
    \end{itemize}
    
    \subsubsection*{Objetivos}
    El análisis de algoritmos tiene dos objetivos:
    \begin{itemize}
        \item Estudiar cuándo y por qué los algoritmos son \textbf{correctos} (es decir, hacen lo que dicen que hacen).
        \item Estimar la cantidad de \textbf{recursos} computacionales que un algoritmo necesita para su ejecución.
    \end{itemize}
    
    \subsubsection*{Pseudo-código}
    Se usa pseudo-código para escribir algoritmos:
    \begin{itemize}
        \item Instrucciones usuales como \textbf{if}, \textbf{while}, \textbf{return}, ...
        \item Notaciones cómodas para arreglos, conjuntos, etc.
    \end{itemize}
    
    Consideraremos que los algoritmos tienen:
    \begin{itemize}
        \item \textbf{Precondiciones:} representan el input del programa.
        \item \textbf{Postcondiciones:} representan el output del programa, lo que hace el algoritmo con el input.
    \end{itemize}
    
    \subsubsection*{Corrección de algoritmos}
    Un algoritmo es
    \begin{itemize}
        \item \textbf{correcto}, si para todo input válido, el algoritmo se detiene y produce un output correcto.
        \item \textbf{incorrecto}, si existe un input válido para el cual el algoritmo no se detiene o produce un output incorrecto.
    \end{itemize}
    
    \subsubsection*{Algoritmos iterativos}
    Debemos demostrar dos cosas:
    \begin{itemize}
        \item \textbf{Corrección parcial:} si el algoritmo se detiene, se cumplen las postcondiciones.
        \item \textbf{Terminación:} el algoritmo se detiene.
    \end{itemize}
    
    \subsubsection*{Demostración}
    Para demostrar la corrección parcial, buscamos un \textbf{invariante} $I(k)$ para los loops:
    \begin{itemize}
        \item Una propiedad $I$ que sea verdadera en cada paso $k$ de la iteración.
        \item Debe relacionar a las variables presentes en el algoritmo.
        \item Al finalizar la ejecución, debe asegurar que las postcondiciones se cumplan.
    \end{itemize}
    
    Una vez encontramos una invariante, demostramos la corrección del loop inductivamente:
    \begin{itemize}
        \item \textbf{Base:} las precondiciones deben implicar que $I(0)$ es verdadero.
        \item \textbf{Inducción:} para todo natural $k > 0$, si $G$ e $I(k)$ son verdaderos antes de la iteración, entonces $I(k + 1)$ es verdadero después de la iteración.
        \item \textbf{Corrección:} inmediatamente después de terminado el loop (i.e. cuando $G$ es falso), si $k = N$ e $I(N)$ es verdadero, entonces las postcondiciones se cumplen.
    \end{itemize}
    
    Y para \textbf{demostrar terminación}, debemos mostrar que existe un $k$ para el cual $G$ es falso.
    
    \subsubsection*{Algoritmos recursivos}
    En el caso de los algoritmos recursivos, no necesitamos dividir la demostración en corrección parcial y terminación.
    \begin{itemize}
        \item Basta con demostrar por inducción la propiedad (corrección) deseada.
        \item En general, la inducción se realiza sobre el tamaño del input.
    \end{itemize}
    
    \subsubsection*{Notación asintótica}
    Lo que nos interesa luego de determinar cuando un algoritmo es correcto, es su comportamiento a medida que crece el \textbf{input}, ya que el hecho de que un algoritmo sea correcto no implica que sea útil en la práctica: hay que determinar su tiempo de ejecución. \p
    
    Vamos a ocupar funciones de dominio natural $(\mathbb{N})$ y recorrido real positivo $(\mathbb{R}^+)$
    \begin{itemize}
        \item El dominio será el tamaño del input de un algoritmo.
        \item El recorrido será el tiempo necesario para ejecutar el algoritmo.
    \end{itemize}
    Sea $f: \mathbb{N} \rightarrow \mathbb{R}^+$, entonces
    \begin{align*}
        O(f) = \{ g: \mathbb{N} \rightarrow \mathbb{R}^+ \,&|\, ( \exists c \in \mathbb{R}^+ ) ( \exists n_0 \in \mathbb{N} ) ( \forall n \geq n_0 ): \\
        &g(n) \leq c \cdot f(n) \}
    \end{align*}
    
    Diremos que $g \in O(f)$ es a lo más de orden $f$ o que es $O(f)$.
    \begin{align*}
        \Omega(f) = \{ g: \mathbb{N} \rightarrow \mathbb{R}^+ \,&|\, ( \exists c \in \mathbb{R}^+ ) ( \exists n_0 \in \mathbb{N} ) ( \forall n \geq n_0 ): \\
        &g(n) \geq c \cdot f(n) \}
    \end{align*}
    
    Diremos que $g \in \Omega(f)$ es al menos de orden $f$ o que es $\Omega(f)$.
    $$
    \Theta(f) = O(f) \cap \Omega(f)
    $$
    Diremos que $g \in \Theta(f)$ es exactamente de orden $f$ o que es $\Theta(f)$.
    
    \subsubsection*{Teorema 21}
    Si $f(n) = a_k \cdot n^k + a_{k - 1} \cdot n^{k - 1} + \ldots a_2 \cdot n^2 + a_1 \cdot n + a_0$ con $a_i \in \mathbb{R}$ y $a_k > 0$, entonces $f$ es $\Theta(n^k)$.
    
    \subsubsection*{Teorema 22}
    Si $f(n) = \log_a(n)$ con $a > 1$, entonces para todo $b > 1$ se cumple que $f$ es $\Theta(\log_b(n))$.
    
    \subsubsection*{Funciones más usadas}
    Las funciones más usadas para los órdenes de notación asintótica tienen nombres típicos.
    
    \begin{table}[H]
        \begin{tabular}{l|l}
         Notación & Nombre  \\
         \hline
         $\Theta(1)$ & Constante \\
         $\Theta(\log(n))$ & Logarítmico  \\
         $\Theta(n)$ & Lineal  \\
         $\Theta(n \log(n))$ & $n \log(n)$  \\
         $\Theta(n^2)$ & Cuadrático  \\
         $\Theta(n^3)$ & Cúbico  \\
         $\Theta(n^k)$ & Polinomial \\
         $\Theta(m^k)$ & Exponencial \\
         $\Theta(n!)$ & Factorial 
        \end{tabular}
        \end{table}
    
    con $k \geq 0$ y $m \geq 2$.
    
    \subsubsection*{Complejidad algoritmos iterativos}
    Queremos encontrar una función $T(n)$ que modele el tiempo de ejecución de un algoritmo.
    \begin{itemize}
        \item Donde $n$ es el tamaño del input.
        \item No queremos valores exactos de $T$ para cada $n$, sino que una notación asintótica para ella.
        \item Para encontrar $T$, contamos las instrucciones ejecutadas por el algoritmo.
        \item A veces contaremos cierto tipo de instrucciones que son relevantes para un algoritmo particular.
    \end{itemize}
    
    Consideremos el siguiente algoritmo de búsqueda de arreglos:
    $$
    \text{\textit{BÚSQUEDA}}(A,n,k)
    $$
    
    Donde:
    \begin{itemize}
        \item \textbf{Input:} un arreglo de $A = [a_0, \ldots, a_{n-1}]$, un natural $n > 0$ correspondiente al largo del arreglo $A$ y un entero $k$.
        \item \textbf{Output:} el índice de $k$ en $A$, $-1$ si no está.
    \end{itemize}
    
    \begin{lstlisting}[numbers=left, numbersep=-10pt, mathescape]
        for $i = 0$ to $n - 1$ do
            if $a_i$ = $k$ then
                return $i$
        return $-1$
    \end{lstlisting}
    
    ¿Qué instrucción(es) contamos?
    \begin{itemize}
        \item Deben ser representativas de lo que hace el problema.
        \item En este caso, por ejemplo 3 y 4 no lo son.
        \item La instrucción 2 si lo sería, y más específicamente la comparación.
        \begin{itemize}
            \item Las comparaciones estan entre las instrucciones que se cuentan típicamente, sobre todo en búsqueda y ordenación.
        \end{itemize}
    \end{itemize}
    
    ¿Respecto a qué parámetro buscamos la notación asintótica?
    \begin{itemize}
        \item En el ejemplo, es natural pensar en el tamaño de arreglo $n$.
    \end{itemize}
    
    \textbf{En conclusión:} queremos encontrar una notación asintótica (ojalá $\Theta$) para la cantidad de veces que se ejecuta la comparación de la línea 2 en función de $n$. Llamaremos a esta cantidad $T(n)$. \p
    
    Ahora, ¿$T(n)$ depende solo de $n$?
    \begin{itemize}
        \item El contenido del arreglo influye en la ejecución del algoritmo.
        \item Estimaremos entonces el tiempo para el \textbf{peor caso} (cuando el input hace que el algoritmo se demore la mayor cantidad de tiempo posible) y el \textbf{mejor caso} (lo contrario) para un tamaño de input $n$.
    \end{itemize}
    
    En nuestro ejemplo:
    \begin{itemize}
        \item \textbf{Mejor caso:} $a_0 = k$. Aquí la línea 2 se ejecuta una vez, y luego $T(n)$ es $\Theta(1)$.
        \item \textbf{Peor caso:} $k$ no esta en $A$. La línea 2 se ejecutará tantas veces como elementos en $A$, y entonces $T(n)$ es $\Theta(n)$.
        \item Diremos entonces que el algoritmo \textit{BÚSQUEDA}$(A,n,k)$ es de \textbf{complejidad} $\Theta(n)$ o lineal en el peor caso y $\Theta(1)$ o constante en el mejor caso.
    \end{itemize}
    
    En general, nos conformaremos con encontrar la complejidad del peor caso.
    \begin{itemize}
        \item Es la que más interesa, al decirnos qué tan mal se puede comportar un algoritmo en la práctica.
    \end{itemize}
    
    Además, a veces puede ser difícil encontrar una notación $\Theta$.
    \begin{itemize}
        \item Nos basta y es suficiente con una buena estimación $O$, tanto para el mejor y el peor caso.
        \item Nos da una cota superior para el tiempo de ejecución del algoritmo.
    \end{itemize}
    
    \subsubsection*{Complejidad algoritmos recursivos}
    En este caso, el principio es el mismo: contar instrucciones.
    \begin{itemize}
        \item Buscamos alguna(s) instrucción(es) representativa(s).
        \item Contamos cuántas veces se ejecuta en cada ejecución del algoritmo.
    \end{itemize}
    
    Tenemos que considerar las llamadas recursivas del algoritmo.
    \begin{itemize}
        \item Esto hara que aparezcan fórmulas recursivas que deberemos resolver.
    \end{itemize}
    
    Por ejemplo, podemos encontrar una función $T(n)$ para la cantidad de comparaciones que realiza un algoritmo en el peor caso, en función del tamaño del arreglo. Este tipo de funciones se denomina \textbf{ecuación de recurrencia}.
    
    
    \subsubsection*{Notación asintótica condicional}
    Sea $P \subseteq \mathbb{N}$. Luego,
    \begin{align*}
        O(f \,|\, P) = \{ &g: \mathbb{N} \rightarrow \mathbb{R}^+ \,|\, (\exists c \in \mathbb{R}^+)(\exists n_0 \in \mathbb{N})\\
        &(\forall n \geq n_0) (n \in P \rightarrow g(n) \leq c \cdot f(n))\}
    \end{align*}
    
    Las notaciones $\Omega(f \,|\,P)$ y $\Theta(f \,|\, P)$ se definen de manera análoga.
    
    \subsubsection*{Función asintóticamente no decreciente}
    Una función $f: \mathbb{N} \rightarrow \mathbb{R}^+$ se denomina así si cumple que
    $$
    (\exists n_0 \in \mathbb{N})(\forall n \geq n_0)(f(n) \leq f(n + 1))
    $$
    Por ejemplo, las funciones $\log_2(n)$, $n$, $n^k$ y $2^n$ son asíntoticamente no decrecientes.
    
    \subsubsection*{Función $b$-armónica}
    Dado un natural $b > 0$, una función $f: \mathbb{N} \rightarrow \mathbb{R}^+$ se denomina $b$-armónica si $f(b \cdot n) \in O(f)$.
    
    \subsubsection*{Teorema 23}
    Sean $f,g: \mathbb{N} \rightarrow \mathbb{R}^+$, un natural $b > 1$ y
    $$
    POTENCIA_b = \{ b^i \,|\, i \in \mathbb{N}\}
    $$
    
    Si $f,g$ son asintóticamente no decrecientes, $g$ es $b$-armónica y $f \in O(g \,|\, POTENCIA_b)$, entonces $f \in O(g)$.
    
    \subsubsection*{Teorema 24 (Maestro)}
    Si $a_1, a_2, b, c, c_0, d \in \mathbb{R}^+$ y $b > 1$, entonces para una recurrencia de la forma
    $$
    T(n) = \begin{cases}
        c_0 &0 \leq n < n_0 \\
        a_1 \cdot T(\lceil\frac{n}{b}\rceil) + a_2 \cdot T(\lfloor\frac{n}{b}\rfloor) + c \cdot n^d &n \geq n_0
    \end{cases}
    $$
    se cumple que
    $$
    T(n) \in \begin{cases}
        \Theta(n^d) &a_1 + a_2 < b^d \\
        \Theta(n^d \cdot \log(n)) &a_1 + a_2 = b^d \\
        \Theta(n^{\log_b(a_1 + a_2)}) &a_1 + a_2 > b^d
    \end{cases}
    $$
    
    \section{Teoría de grafos}
    
    \subsubsection*{Grafo}
    Un grafo $G = (V,E)$ es un par donde $V$ es un conjunto, cuyos elementos llamaremos \textit{vértices o nodos}, y $E$ es una relación binaria sobre $V$ (es decir, $E \subseteq V \times V$), cuyos elementos llamaremos \textit{aristas}. \p
    
    Para representar un grafo, usamos puntos o círculos para dibujar vértices, y flechas para dibujar aristas. Cada arista será una flecha entre los nodos que relaciona. Estos son los llamados \textbf{grafos dirigidos}.
    
    \subsubsection*{Rulo (loop)}
    Es una arista $(x,y) \in E$ tal que $x = y$. Es decir, es una arista que conecta un vértice con sí mismo.
    
    \subsubsection*{Aristas paralelas}
    Dos aristas $(x,y) \in E$  y $(z, w) \in E$ son pararelas si $x = w$ e $y = z$. Es decir, si conectan a los mismos vértices.
    
    \subsubsection*{Grafo no dirigido}
    Un grafo $G = (V, E)$ es no dirigido si toda arista tiene una arista paralela. Alternativamente, $G$ es no dirigido si $E$ es simétrica. En estos grafos se dibuja con trazos en lugar de flechas.
    
    \subsubsection*{Grafo simple}
    Un grafo \textbf{no dirigido} $G = (V, E)$ es simple si no tiene rulos. Alternativamente, $G$ es simple si $E$ es irrefleja.
    
    \subsubsection*{Convención sobre grafos}
    De ahora en adelante (a menos que se explicite otra cosa), cuando hablemos de grafos estaremos refiriéndonos a grafos \textbf{simples}, \textbf{no dirigidos}, \textbf{no vacíos} y \textbf{finitos}.
    \begin{itemize}
        \item $V \neq \varnothing$ y $|V| = n$, con $n \in \mathbb{N}$.
        \item $E$ es simétrica e irrefleja.
    \end{itemize}
    
    \subsubsection*{Vértices adyacentes o vecinos}
    Dado un grafo $G = (V, E)$, dós vértices $x,y \in V$ son adyacengtes o vecinos si $(x,y) \in E$.
    
    \subsubsection*{Isomorfismo}
    Dos grafos $G_1 = (V_1, E_1)$ y $G_2 = (V_2, E_2)$ son \textbf{isomorfos} si existe una función biyectiva $f: V_1 \rightarrow V_2$ tal que $(x,y) \in E_1$ si y sólo si $(f(x), f(y)) \in E_2$. En tal caso:
    \begin{itemize}
        \item Diremos que $f$ es un \textbf{isomorfismo} entre $G_1$ y $G_2$.
        \item Escribiremos $G_1 \cong G_2$.
    \end{itemize}
    
    \subsubsection*{Teorema 25}
    La relación de isomorfismo $\cong$ es una relación de equivalencia (es decir, es refleja, simétrica y transitiva).
    
    \subsubsection*{Camino (informal)}
    Un camino es un grafo cuyos vértices pueden dibujarse en una línea tal que dos vértices son adyacentes si y sólo si aparecen consecutivos en la línea.
    
    \subsubsection*{Camino (formal)}
    Considere un grafo $G_n^P = (V_n^P, E_n^P)$, donde 
    \begin{align*}
        V_n^P &= \{ v_1, \ldots, v_n \} \\ 
        E_n^P &= \{ (v_i, v_j) \,|\, i \in \{1,\ldots, n - 1\} \wedge j = i + 1\}
    \end{align*}
    
    Un \textbf{camino} (de $n$ vértices) es un grafo isomorfo a $G_n^P$. \p
    
    A la clase de equivalencia $[G_n^P]_{\cong}$ la llamaremos $P_n$: los caminos con $n$ vértices.
    
    \subsubsection*{Ciclo (informal)}
    Un ciclo es un grafo cuyos vértices pueden dibujarse en un círculo tal que dor vértices son adyancentes si y sólo si aparecen consecutivos en él.
    
    \subsubsection*{Ciclo (formal)}
    Considere un grafo $G_n^C = (V_n^C, E_n^C)$, donde 
    \begin{align*}
        V_n^C &= \{ v_1, \ldots, v_n \} \\ 
        E_n^C &= \{ (v_i, v_j) \,|\, i \in \{1,\ldots, n - 1\} \wedge j = i + 1\} \cup \{ (v_n, v_1) \}
    \end{align*}
    
    Un \textbf{ciclo} (de $n$ vértices) es un grafo isomorfo a $G_n^P$. \p
    
    A la clase de equivalencia $[G_n^C]_{\cong}$ la llamaremos $C_n$: los ciclos con $n$ vértices.
    
    \subsubsection*{Grafo completo}
    Un grafo completo es un grafo en el que todos los pares de vértices son adyacentes. \p
    
    A la clase de equivalencia de los grafos completos de $n$ vértices la llamaremos $K_n$.
    
    \subsubsection*{Grafo bipartito}
    Un grafo $G = (V,E)$ se deice bipartito si $V$ se puede particionar en dos conjuntos no vacíos $V_1$ y $V_2$ tales que para toda arista $(x,y) \in E$, $x \in V_1$ e $y \in V_2$, o $x \in V_2$ e $y \in V_1$. \p
    
    Es decir,
    \begin{itemize}
        \item $V = V_1 \cup V_2$
        \item $V_1 \cap V_2 = \varnothing$
        \item Cada arista une a dos vértices en conjuntos distintos de la partición.
    \end{itemize}
    
    \subsubsection*{Grafo bipartito completo}
    Un grafo bipartito completo es un grafo bipartito en que cada vértice es adyacente a todos los de la otra partición. \p
    
    A la clase de equivalencia de los grafos bipartitos completos la llamaremos $K_{n,m}$, donde $n$ y $m$ son los tamaños de las particiones.
    
    \subsubsection*{Subgrafo}
    Dado un grafo $G = (V_G, E_G)$, un grafo $H = (V_H, E_H)$ es un subgrafo de $G$ (denotado como $H \subseteq G$) si $V_H \subseteq V_G$, $E_H \subseteq E_G$ y $E_H$ sólo contiene aristas entre vértices de $V_H$.
    
    \subsubsection*{Clique}
    Dado un grafo $G = (V_G, E_G)$, un clique en $G$ es un conjunto de vértices $K \subseteq V_G$ tal que
    $$\forall\,v_1, v_2 \in K.\,(v_1, v_2) \in E_G$$
    
    \subsubsection*{Conjunto independiente}
    Dado un grafo $G = (V_G, E_G)$, un conjunto independiente en $G$ es un conjunto de vértices $K \subseteq V_G$ tal que
    $$
    \forall\,u,v \in K.\, (u,v) \notin E_G
    $$
    
    \subsubsection*{Complemento}
    Dado un grafo $G = (V_G, E_G)$, el complemento de $G$ es el grafo $\overline{G} = (V_G, \overline{E_G})$, donde $(u,v) \in E_G \leftrightarrow (u,v) \notin \overline{E_G}$.
    
    \subsubsection*{Grafo autocomplementario}
    Un grafo $G$ se dice autocomplementario si $G \cong \overline{G}$.
    
    \subsubsection*{Teorema 26}
    Dado un grafo $G = (V,E)$, un conjunto $V' \subseteq V$ es un clique en $G$ si y sólo si es un conjunto independiente en $\overline{G}$.
    
    \subsubsection*{Matriz de adyacencia}
    Dado un grafo $G = (V,E)$, como $E$ es una relación binaria podemos representarla en una matriz, llamada matriz de adyacencia de $G$. \p
    
    Por ejemplo, si $V = \{1,2,3,4\}$ y $E = \{ (1,2),$
    $(1,4), (2,1), (2,3), (2,4), (3,2), (3,4), (4,1), (4,2), (4,3) \}$, entonces
    $$
    M_G = \begin{bmatrix}
        0 & 1 & 0 & 1 \\
        1 & 0 & 1 & 1 \\
        0 & 1 & 0 & 1 \\
        1 & 1 & 1 & 0 
        \end{bmatrix}
    $$
    
    \begin{itemize}
        \item Si el grafo es simple, la diagional sólo contiene ceros.
        \item Si el grafo es no dirigido, entonces $M_G = M_G^T$.
    \end{itemize}
    
    \subsubsection*{Matriz de incidencia}
    Además de la matriz de adyacencia, podemos usar una matriz de incidencia $A_G$:
    \begin{itemize}
        \item Etiquetamos las aristas de $G$.
        \item Cada fila de la matriz representará un vértice, y cada columna a una arista.
        \item Cada posición de la matriz tendra un 1 si la arista de la columna \textit{incide} en el vértice de al fila.
    \end{itemize}
    
    Por ejemplo, si $V = \{1,2,3,4\}$ y $E = \{ (1,2),$
    $(1,4), (2,1), (2,3), (2,4), (3,2), (3,4), (4,1), (4,2), (4,3) \}$, entonces
    $$
    A_G = \begin{bmatrix}
        1 & 1 & 0 & 0 & 0 \\
        1 & 0 & 1 & 1 & 0\\
        0 & 0 & 1 & 0 & 1\\
        0 & 1 & 0 & 1 & 1
        \end{bmatrix}
    $$
    
    \subsubsection*{Grado de un grafo}
    Dado un grafo $G$ y un vértice $v$ de él, el grado de $v$ (denotado como $\delta_G(v)$) es la cantidad de aristas que inciden en $v$.
    
    \subsubsection*{Vecindad de un grafo}
    Dado un grafo $G$ y un vértice $v$ de él, la vecindad de $v$ es el conjunto de vecinos de $v$:
    $$
    N_G(v) = \{ u \,|\, (v,u) \in E \}
    $$
    
    En un grafo simple, $\delta_G(v) = |N_G(v)|$
    
    \subsubsection*{Teorema 27 (Handshaking lemma)}
    Si $G = (V,E)$ es un grafo sin rulos, entonces
    $$
    \sum_{v \in V} \delta_G(v) = 2|E|
    $$
    Es decir, la suma de los grafos de los vértices es dos veces la cantidad de aristas.
    
    \begin{itemize}
        \item \textbf{Corolario:} En un grafo sin rulos siempre hay una cantidad par de vértices de grado impar.
    \end{itemize}
    
    \subsubsection*{Caminatas}
    Una caminata en un grafo $G = (V,E)$ es una secuencia de vértices $(v_0, v_1, v_2, \ldots, v_k)$, con $v_0, \ldots, v_k \in V$, tal que $(v_{i - 1}, v_i) \in E$, con $i$ entre 1 y $k$. \p
    
    Una caminata \textbf{cerrada} en un grafo es una caminata que empieza y termina en el mismo vértice: $v_0 = v_k$.
    
    \subsubsection*{Caminos y ciclos}
    Un \textbf{camino} es un grafo es una caminata en la que no se repiten aristas. \p
    
    Un \textbf{ciclo} en un grafo es una caminata cerrada en la que no se repiten aristas.
    
    \subsubsection*{Largo de una caminata, camino o ciclo}
    Corresponde a la cantidad de aristas que lo componen. Si está compuesto por un único vértice (sin aristas), diremos que tiene largo 0.
    
    \subsubsection*{Vértices conectados}
    Dos vértices $x$ e $y$ en un grafo $G$ estan conectados si existe un camino en $G$ que empieza en $x$ y termina en $y$. \p
    
    \textbf{Nota:} Que dos vértices ``estén conectados'' implica una relación de equivalencia.
    
    \subsubsection*{Componente conexa}
    Dado un vértice $v$ de un grafo $G$, su clase de equivalencia bajo la relación ``estar conectados'' es una componente conexa de $G$. \p
    
    En general, diremos que la componente conexa tambien contiene a las aristas entre los vértices de ella.
    
    \subsubsection*{Grafo conexo}
    Un grafo $G$ se dice conexo si todo par de vértices $x,y \in V$ está conectado. En otro caso, $G$ es disconexo. \p
    
    En otras palabras, esta definición es equivalente a decir que $G$ tiene sólo una componente conexa.
    
    \subsubsection*{Teorema 28}
    Un grafo $G$ con $n$ vértices y $k$ aristas tiene al menos $n - k$ componentes conexas.
    \begin{itemize}
        \item \textbf{Corolario:} Si un grafo $G$ con $n$ vértices es conexo, tiene al menos $n - 1$ aristas.
    \end{itemize}
    
    \subsubsection*{Arista de corte}
    Una arista de corte en un grafo $G$ es una arista tal que al eliminarla aumenta la cantidad de componentes conexas de $G$.
    
    \subsubsection*{Vértice de corte}
    Un vértice de corte en un grafo $G$ es un vértice tal que al eliminarlo (junto con todas sus aristas incidentes) aumenta la cantidad de componentes conexas de $G$.
    
    \subsubsection*{Teorema 29}
    Una arista en un grafo $G$ es de corte si y sólo si no pertenece a ningún ciclo en $G$.
    
    \subsubsection*{Lema 2}
    En un grafo simple $G$, toda caminata cerrada de largo impar contiene un ciclo de largo impar.
    
    \subsubsection*{Teorema 30}
    Un grafo simple conexo $G$ es bipartito si y sólo si no contiene ningún ciclo de largo impar.
    
    \subsubsection*{Multigrafo}
    Sea $V$ un conjunto de vértices, $E$ un conjunto de aristas y $S \subseteq \mathcal{P}(V)$ tal que
    $$
    S = \{ \{u,v\} \,|\, u \in V \wedge v \in V\}
    $$
    
    Un multigrafo $G = (V, E, f)$ es un trío ordenado donde $f: E \rightarrow S$ es una función que asigna un par de vértices a cada arista en $E$.
    
    \subsubsection*{Ciclo Euleriano}
    Un ciclo euleriano en un (multi)grafo $G$ es un ciclo que contiene a todas las aristas y vértices de $G$.
    \begin{itemize}
        \item Es un ciclo, por lo tanto no puede repetir aristas.
        \item Pueden repetirse vértices.
        \item Iremos que $G$ es un \textbf{grafo Euleriano} si contiene un ciclo Euleriano.
    \end{itemize}
    
    \subsubsection*{Teorema 31}
    Un (multi)grafo sin rulos es Euleriano si y sólo si es conexo y todos sus vértices tienen grado par.
    
    \subsubsection*{Camino Euleriano}
    Un camino Euleriano en un (multi)grafo $G$ es un camino no cerrado que contiene a todas las aristas y vértices de $G$.
    
    \subsubsection*{Teorema 32}
    Un (multi)grafo tiene un camino Euleriano si y sólo si es conexo y contiene exactamente dos vértices de grado impar.
    
    \subsubsection*{Ciclo Hamiltoniano}
    Un ciclo Hamiltoniano en un grafo $G$ es un ciclo en $G$ que contiene a todos sus vértices una única vez cada uno (excepto por el inicial y el final).
    \begin{itemize}
        \item Diremos que $G$ es un \textbf{grafo Hamiltoniano} si contiene un ciclo Hamiltoniano.
        \item No hay ninguna relación entre grafos Eulerianos y Hamiltonianos.
    \end{itemize}
    
    \subsubsection*{Árbol}
    Un grafo $T = (V, E)$ es un \textbf{árbol} si para cada par de vértices $x,y \in V$ existe un único camino entre ellos. Por lo tanto, siempre es conexo.
    
    \subsubsection*{Bosque}
    Un grafo $T = (V, E)$ es un \textbf{bosque} si para cada par de vértices $x,y \in V$, si existe un camino entre ellos, este es único. Un bosque es un conjunto de árboles.
    
    \subsubsection*{Árboles con raíz}
    Distinguimos uno de los vértices $r \in V$, al que llamaremos la \textbf{raíz} del árbol. Los vértices de grado menor o igual a 1 se llaman \textbf{hojas}. Los dibujamos con la raíz arriba y los demás vértices hacia abajo.
    
    \subsubsection*{Definiciones alternativas de árbol}
    Hay muchas definiciones equivalentes para los árboles:
    \begin{enumerate}
        \item Un grafo $T = (V, E)$ es un \textbf{árbol} si y sólo si es conexo y acíclico.
        \item Un grafo $T = (V, e)$ es un \textbf{árbol} si y sólo si es conexo y todas sus aristas son de corte.
    \end{enumerate}
    
    \subsubsection*{Teorema 33}
    Todo árbol es un grafo bipartito.
    
    \subsubsection*{Teorema 34}
    Si $T$ es un árbol y $v$ es una hoja de él, entonces el grafo $T - v$ (el grafo que resulta de quitar el vértice y sus aristas incidentes) es un árbol.
    
    \subsubsection*{Otra definición útil de árbol}
    Un grafo $T = (V, E)$ con $n$ vértices es un \textbf{árbol} si y sólo si es conexo y tiene exactamente $n - 1$ aristas.
    
    \subsubsection*{Profundidad, altura, ancestros, padre, hijos y hermanos}
    Sea $T = (V, E)$ un árbol con raíz $r$ y $x$ un vértice cualquiera.
    \begin{itemize}
        \item La \textbf{profundidad} de $x$ es el largo del camino que lo une con $r$ ($r$ tiene profundidad 0).
        \item La \textbf{altura} o \textbf{profundidad} del árbol es el máximo de las profundidades de sus vértices.
        \item Los \textbf{ancestros} de $x$ son los vértices que aparecen en el camino entre él y $r$. Note que $x$ es ancestro de sí mismo.
        \item El \textbf{padre} de $x$ es su ancestro (propio) de mayor profundidad. Diremos que $x$ es \textbf{hijo} de su padre.
        \item Dos vértices $x$ e $y$ con el mismo padre son \textbf{hermanos}.
    \end{itemize}
    
    \subsubsection*{Árboles binarios}
    Un árbol con raíz se dice \textbf{binario} si todo vértice tiene grafo a lo más 3; o equivalentemente, si todo vértice tiene a lo más dos hijos. Podemos distinguir entre hijos izquierdos y derechos.
    
    \subsubsection*{Teorema 35}
    La cantidad de vértices sin hijos de un árbol binario es la cantidad de vértices con exactamente dos hijos más 1.
    
    \subsubsection*{Árbol binario completo}
    Un \textbf{árbol binario completo} es un árbol binario tal que:
    \begin{enumerate}
        \item Todas las hojas están a la misma profundidad.
        \item Todos los vértices que no son hojas tienen exactamente dos hijos.
    \end{enumerate}
    
    \subsubsection*{Teorema 36}
    \begin{enumerate}
        \item Un árbol binario completo de altura $H$ tiene exactamente $2^H$ hojas.
        \item Un árbol binario completo de altura $H$ tiene exactamente $2^{H + 1} - 1$ vértices.
        \item Si $H$ es la altura de un árbol binario completo con $n$ vértices, entonces $H \leq \log_2(n)$.
    \end{enumerate}
    
    \section{Teoría de números}
    \subsubsection*{Recordando la relación $|$}
    La relación \textit{divide a}, denotada por $|$, sobre los enteros sin el 0, es una relación tal que $a$ está relacionado con $b$ si y sólo $b$ es múltiplo de $a$:
    $$
    a|b \text{ si y sólo si } \exists k \in \mathbb{Z} \text{ tal que } b = ka
    $$
    
    \subsubsection*{Recordando la relación $\equiv_n$}
    La relación \textit{equivalencia módulo} $n$, denotada por $\equiv_n$, sobre los enteros, es una relación tal que $a$ está relacionado con $b$ si y sólo si $n|(b - a)$:
    \begin{align*}
        &a \equiv_n b \text{ si y sólo si } n|(b - a) \\
        a \equiv_n b &\text{ si y sólo si } \exists k \in \mathbb{Z} \text{ tal que } (b - a) = kn
    \end{align*}
    \begin{itemize}
        \item La relación $\equiv_n$ es una relación de equivalencia.
        \item Podemos tomar el conjunto cuociente generado por ella sobre $\mathbb{Z}$.
        \item Usando las clases de equivalencia, definimos la suma y la multiplicación.
    \end{itemize}
    
    \subsubsection*{Suma y multiplicación}
    Dado $n \in \mathbb{N}, n > 0$, definimos 
    $$
    \mathbb{Z}_n = \mathbb{Z} / \equiv_n
    $$
    y sus operaciones
    \begin{align*}
        [i] + [j] &= [i + j] \\
        [i] \cdot [j] &= [i \cdot j]
    \end{align*}
    
    \subsubsection*{Operación módulo $n$}
    La operación \textbf{módulo} $n$ entrega el resto de la división por $n$. Se escribe $a$ mod $n$ o $a \% n$. \p
    
    Con esta operación podemos redefinir la suma y multiplicación en $\mathbb{Z}_n$:
    \begin{align*}
        [i] + [j] &= (i + j) \text{ mod } n\\
        [i] \cdot [j] &= (i \cdot j) \text{ mod } n
    \end{align*}
    \begin{itemize}
        \item Una observación importante es que siempre se cumple que
        $$
        0 \leq a \text{ mod } n < n
        $$
    \end{itemize}
    
    \subsubsection*{Teorema 37}
    $a \equiv_n b$ si y sólo si $a \text{ mod } n = b \text{ mod } n$.
    \begin{itemize}
        \item \textbf{Corolario:} $a \equiv_n a \text{ mod } n$.
    \end{itemize}
    
    \subsubsection*{Teorema 38}
    Si $a \equiv_n b$ y $c \equiv_n d$, entonces
    \begin{align*}
        (a + c) &\equiv_n (b + d) \\
        (a \cdot c) &\equiv_n (b \cdot d)
    \end{align*}
    \begin{itemize}
        \item \textbf{Corolario:}
        \begin{align*}
            (a + b) \text{ mod } n &= ((a \text{ mod } n) + b \text{ mod } n) \text{ mod } n \\
            a \cdot b \text{ mod } n &= ((a \text{ mod } n)(b \text{ mod } n)) \text{ mod } n \\
        \end{align*}
    \end{itemize}
    
    \subsubsection*{Teorema 39 (Fermat)}
    Si $p$ es un numero primo, para cualquier entero $a$ se cumple que $a^p \equiv_p a$.
    \begin{itemize}
        \item \textbf{Corolario:} Si $p$ es un número primo y $a$ es un entero que no es múltiplo de $p$, entonces $a^{p - 1} \equiv_p 1$.
    \end{itemize}
    
    \subsubsection*{Máximo común divisor (informal)}
    Dados dos números $a$ y $b$, su máximo común divisor, denotado como $MCD(a,b)$, es el máximo natural $n$ tal que $n|a$ y $n|b$.
    
    \subsubsection*{Teorema 40}
    Si $b > 0$, entonces $MCD(a,b) = MCD(b, a \text{ mod } b)$.
    $$
    MCD(a,b) = \begin{cases}
    a &b = 0 \\
    MCD(b, a \text{ mod } b) &b > 0
    \end{cases}
    $$
    
    \subsubsection*{Calculando enteros con $MCD$}
    Con el algoritmo (ecuación) anterior, se puede calcular $s,t \in \mathbb{Z}$ sabiendo que
    $$
    MCD(a,b) = s \cdot a + t \cdot b
    $$ 
    
    \subsubsection*{Algoritmo extendido del $MCD$}
    Sea $a \geq b$.
    \begin{enumerate}
        \item Definimos una sucesión $\{ r_i \}$ como:
        $$
        r_0 = a, \, r_1 = b, \, r_{i + 1} = r_{i - 1} \text{ mod } r_i
        $$
        
        \item Definimos sucesiones $\{s_i\}, \{t_i\}$ tales que:
        \begin{align*}
            s_0 &= 1,\,t_0 = 0 \\
            s_1 &= 0, \, t_1 = 1 \\
            r_i &= s_i \cdot a + t_i \cdot b
        \end{align*}
        
        \item Calculamos estas sucesiones hasta un $k$ tal que $r_k = 0$.
        \item Entonces, $MCD(a,b) = r_{k - 1} = s_{k - 1} \cdot a + t_{k - 1} \cdot b$.
    \end{enumerate}
    
    \subsubsection*{Algoritmo extendido del $MCD$ (alternativa)}
    Sea $a \geq b$.
    \begin{enumerate}
        \item Definimos una sucesión $\{ r_i \}$ como:
        $$
        r_0 = a, \, r_1 = b, \, r_{i + 1} = r_{i - 1} \text{ mod } r_i
        $$
        
        \item Definimos sucesiones $\{s_i\}, \{t_i\}$ tales que:
        \begin{align*}
            s_0 &= 1,\,t_0 = 0 \\
            s_1 &= 0, \, t_1 = 1 \\
            s_{i + 1} = s_{i - 1} - \left\lfloor \dfrac{r_{i-1}}{r_i} \right\rfloor &\cdot s_i,\; t_{i + 1} = t_{i - 1} - \left\lfloor \dfrac{r_{i-1}}{r_i} \right\rfloor \cdot t_i
        \end{align*}
        
        \item Calculamos estas sucesiones hasta un $k$ tal que $r_k = 0$.
        \item Entonces, $MCD(a,b) = r_{k - 1} = s_{k - 1} \cdot a + t_{k - 1} \cdot b$.
    \end{enumerate}
    
    \subsubsection*{Inverso}
    $b$ es inverso de $a$ en módulo $n$ si $a \cdot b \equiv_n 1$. Se puede denotar como $a^{-1}$. Ojo: no es lo mismo que $\frac{1}{a}$.
    
    \subsubsection*{Teorema 41}
    $a$ tiene inverso en módulo $n$ si y sólo si $MCD(a,n) = 1$. Si se cumple esto, diremos que $a$ y $n$ son \textbf{primos relativos} o \textbf{coprimos}.
    
    \subsubsection*{Método para calcular el inverso}
    La demostración del teorema 41 nos da un método para calcular el inverso:
    \begin{itemize}
        \item Usamos el algoritmo extendido del máximo común divisor para encontrar $s$ y $t$ tales que $1 = s \cdot a + t \cdot n$.
        \item $s$ será el inverso de $a$ en módulo $n$.
    \end{itemize}
    
    Dados $a,b,n \in \mathbb{Z}$, si $a \equiv_n b$ tambien podemos escribir:
    $$
    a \equiv b \text{ (mod } n)
    $$
    Esta es la notación más usada en la literatura.
    
    \subsubsection*{Congruencia lineal}
    Una congruencia lineal es una ecuación de la forma
    $$
    ax \equiv b \text{ (mod } n)
    $$
    donde $n \in \mathbb{N} - \{0\}$, $a,b \in \mathbb{Z}$ y $x$ es una variable.
    
    \subsubsection*{Corolario teorema 41}
    Si $a$ y $n$ son primos relativos, entonces $ax \equiv b \text{ (mod } n)$ tiene  solución en $\mathbb{Z}_n$.
    
    \subsubsection*{Teorema 42 (Chino del Resto)}
    Sean $m_1, m_2, \ldots, m_n$ con $m_i > 1$ tal que $m_i,m_j$ son primos relativos con $i \neq j$. Para $a_1, a_2, \ldots, a_n \in \mathbb{Z}$, el sistema de ecuaciones:
    \begin{align*}
        x &\equiv a_1 \text{ (mod } m_1) \\
        x &\equiv a_2 \text{ (mod } m_2) \\
        &\vdots \\
        x &\equiv a_n \text{ (mod } m_n)
    \end{align*}
    tiene una única solución en $\mathbb{Z}_m$ con $m = \displaystyle{\prod_{i=1}^n m_i}$
    
    \subsubsection*{Lema 3}
    Sean $m_1,m_2 > 1$ coprimos y $u,v \in \mathbb{Z}$. Si $u \equiv v \text{ (mod } m_1)$ y $u \equiv v \text{ (mod } m_2)$, entonces 
    $$u \equiv v \text{ (mod } m_1 \cdot m_2)$$
    
    Este lema se puede generalizar para $n$ coprimos.
    
    \section{Complejidad computacional}
    \subsubsection*{Problemas de decisión (informal)}
    Son los problemas para los cuales sus respuestas posibles son SI o NO.
    
    \subsubsection*{Problemas de decisión (formal)}
    Un problema de decisión $\pi$ se compone de un conjunto de instancias $I_\pi$ y un lenguaje $L_\pi \subseteq I_\pi$, y se define como
    $$
    \text{Dado un elemento } w \in I_\pi, \text{ determinar si } w \in L_\pi
    $$
    Ejemplos:
    \begin{itemize}
        \item PRIMO:
        \begin{itemize}
            \item $I_{\text{PRIMO}} = \mathbb{N}$.
            \item $L_{\text{PRIMO}} = \{ p \in \mathbb{N} \,|\, p \text{ es primo}\}$
        \end{itemize}
        \item EULERIANO:
        \begin{itemize}
            \item $I_{\text{EULERIANO}} = \{ G \,|\, G \text{ es un grafo} \}$
            \item $L_{\text{EULERIANO}} = \{ G \,|\, G \text{ es un grafo con ciclo euleariano} \}$
        \end{itemize}
        \item SAT:
        \begin{itemize}
            \item $I_{\text{SAT}} = L(P)$
            \item $L_{\text{SAT}} = \{ \alpha \in L(P) \,|\, \alpha \text{ es satisfacible} \}$
        \end{itemize}
    \end{itemize}
    
    Un algoritmo $A$ resuelve un problema de decisión $\pi$ si para cada input $w \in I_\pi$, el algoritmo $A$ responde SI cuando $w \in L_\pi$, y responde NO cuando $w \in I_\pi - L_\pi$ (en este caso diremos que $w \notin L_\pi$).
    
    \subsubsection*{Clase de complejidad DTIME}
    La clase de complejidad \textbf{DTIME}$(T(n))$ es el conjunto de problemas para los cuales existe un algoritmo que lo resuelve de complejidad $O(T(n))$, donde $n$ es el tamaño del input: \p
    
    DTIME$(T(n)) = \{ \pi \,|\, \pi$ es un problema de decisión para el cual existe un algoritmo $A$ que lo resuelve y que para todo $w \in I_\pi$ corre en $O(T(|w|))\}$. \p
    
    Los ejemplos anteriores pertenecen a las clases de complejidad:
    \begin{itemize}
        \item PRIMO $\in$ DTIME$(10^n)$
        \item EULERIANO $\in$ DTIME$(n^2)$
        \item SAT $\in$ DTIME$(2^n)$
    \end{itemize}
    
    \subsubsection*{Clase de complejidad PTIME}
    
    La clase de complejidad \textbf{PTIME} o simplemente P se define como
    $$
    \text{P} = \bigcup_{k = 0}^\infty \text{DTIME}(n^k)
    $$
    La clase PTIME contiene a todos los problemas que pueden resolverse en tiempo polinomial. En general diremos que estos problemas son \textit{tratables}. Por ejemplo, EULERIANO $\in$ P.
    
    \subsubsection*{Clase de complejidad NP}
    Como definición informal, la clase de complejidad NP contiene a todos los problemas de decisión para los cuales es posible verificar una solución al problema de forma eficiente. Sin embargo, no necesariamente encontrar la solución es fácil. \p
    
    En una vía más formal, la clase de complejidad NP contiene a todos los problemas de decisión $\pi$ para los cuales se cumple lo siguiente: \p
    
    Si $w \in L_\pi$, entonces existe un \textit{certificado} $c(w)$, de tamaño polinomial respecto a $w$, tal que existe un algoritmo que usando $c(w)$ puede determinar en tiempo polinomial si $w \in L_\pi$.
    
    \subsubsection*{Teorema 43}
    P $\subseteq$ NP. Este teorema nos permite establecer una primera cota para las clases P y NP.
    
    \subsubsection*{Reducciones polinomiales}
    Dados dos problemas de decisión $\pi = (I_\pi, L_\pi)$ y $\pi' = (I_{\pi'}, L_{\pi'})$, diremos que $\pi'$ \textit{se reduce desde} $\pi$ si dada una instancia $w \in I_\pi$ existe una función $A: I_\pi \rightarrow I_{\pi'}$ polinomial en $|w|$ tal que
    $$
    w \in L_\pi \leftrightarrow A(w) \in L_{\pi'}
    $$
    En este caso diremos que $\pi'$ es \textbf{por lo menos tan difícil} como $\pi$ y lo denotaremos como $\pi \propto \pi'$.
    
    \subsubsection*{Fórmula en 3-CNF}
    Decimos que una fórmula $\alpha$ está en 3-CNF si es una conjunción de cláusulas de exactamente 3 literales cada una.
    \begin{itemize}
        \item En general aplicamos la restricción de 3 literales al conjunto que representa a cada cláusula.
        \item Por lo tanto, no pueden repetirse literales en la misma cláusula.
    \end{itemize}
    
    \subsubsection*{SAT-3CNF}
    Definimos SAT-3CNF como
    \begin{align*}
        I_{\text{SAT-3CNF}} &= \alpha \in L(P) \text{ en 3-CNF} \\
        L_{\text{SAT-3CNF}} = \{ \alpha \in L(&P) \text{ en 3-CNF} \,|\, \alpha \text{ es satisfacible} \}
    \end{align*}
    
    En este caso, diremos que SAT-3CNF es \textbf{por lo menos tan difícil} como SAT-3CNF.
    
    \subsubsection*{Teorema 44}
    $\propto$ es un preorden (refleja y transitiva).
    
    \subsubsection*{NP-hard}
    El hecho de que $\propto$ sea un preorden nos permite establecer una jerarquía en cuanto a la dificultad de los problemas. \p
    
    Sea $\pi$ un problema de decisión. Diremos que $\pi$ es NP-hard si para todo $\pi' \in NP$ se tiene que $\pi' \propto \pi$. \p
    
    En otras palabras, los problemas NP-hard son por lo menos tan difíciles como todos los problemas en NP.
    
    \subsubsection*{NP-completo}
    El único problema con la definición de NP-hard es que no nos dice nada sobre las cotas superiores de los problemas. La siguiente definición logra capturar la verdadera complejidad de la clase NP: \p
    
    Sea $\pi$ un problema de decisión. Diremos que $\pi$ es NP-completo si:
    \begin{itemize}
        \item $\pi \in$ NP.
        \item $\pi \in$ NP-hard.
    \end{itemize}
    Dado que cualquier problema en NP se puede reducir a uno NP-completo, podemos concluir que los problemas NP-completos son los más difíciles de la clase NP.
    
    \subsubsection*{Teorema 45 (de Cook)}
    SAT-CNF es NP-completo.
    
    \subsubsection*{Problema de decisión: Clique}
    \begin{align*}
        I_{\text{CLIQUE}} &= \text{Todos los grafos no dirigidos.} \\
        L_{\text{CLIQUE}} = \{ (G, k) \,|\, &G \text{ tiene un clique de tamaño } k \wedge k > 2 \}
    \end{align*}
    
    CLIQUE es NP-completo.
    
    
    \end{multicols}

\section*{Nota al lector}
Este documento está libre tanto para su edición como para su uso o distribución. Si deseas clonar el repositorio de GitHub (\verb|crow-rojas/Resumen-IIC1253-PUC|) y editar el documento a tu manera, eres totalmente bienvenido/a. \\

Son también bienvenidas recomendaciones y/o sugerencias para mejorar aún más el documento, así que si se te ocurre alguna no dudes en escribirme a mi correo \verb|carojas37@uc.cl|. \\

¡Éxito con el estudio!

    \end{document}